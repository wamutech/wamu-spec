% Options for packages loaded elsewhere
\PassOptionsToPackage{unicode}{hyperref}
\PassOptionsToPackage{hyphens}{url}
%
\documentclass[
]{article}
\usepackage{amsmath,amssymb}
\usepackage{iftex}
\ifPDFTeX
  \usepackage[T1]{fontenc}
  \usepackage[utf8]{inputenc}
  \usepackage{textcomp} % provide euro and other symbols
\else % if luatex or xetex
  \usepackage{unicode-math} % this also loads fontspec
  \defaultfontfeatures{Scale=MatchLowercase}
  \defaultfontfeatures[\rmfamily]{Ligatures=TeX,Scale=1}
\fi
\usepackage{lmodern}
\ifPDFTeX\else
  % xetex/luatex font selection
\fi
% Use upquote if available, for straight quotes in verbatim environments
\IfFileExists{upquote.sty}{\usepackage{upquote}}{}
\IfFileExists{microtype.sty}{% use microtype if available
  \usepackage[]{microtype}
  \UseMicrotypeSet[protrusion]{basicmath} % disable protrusion for tt fonts
}{}
\makeatletter
\@ifundefined{KOMAClassName}{% if non-KOMA class
  \IfFileExists{parskip.sty}{%
    \usepackage{parskip}
  }{% else
    \setlength{\parindent}{0pt}
    \setlength{\parskip}{6pt plus 2pt minus 1pt}}
}{% if KOMA class
  \KOMAoptions{parskip=half}}
\makeatother
\usepackage{xcolor}
\setlength{\emergencystretch}{3em} % prevent overfull lines
\providecommand{\tightlist}{%
  \setlength{\itemsep}{0pt}\setlength{\parskip}{0pt}}
\setcounter{secnumdepth}{-\maxdimen} % remove section numbering
\newlength{\cslhangindent}
\setlength{\cslhangindent}{1.5em}
\newlength{\csllabelwidth}
\setlength{\csllabelwidth}{3em}
\newlength{\cslentryspacingunit} % times entry-spacing
\setlength{\cslentryspacingunit}{\parskip}
\newenvironment{CSLReferences}[2] % #1 hanging-ident, #2 entry spacing
 {% don't indent paragraphs
  \setlength{\parindent}{0pt}
  % turn on hanging indent if param 1 is 1
  \ifodd #1
  \let\oldpar\par
  \def\par{\hangindent=\cslhangindent\oldpar}
  \fi
  % set entry spacing
  \setlength{\parskip}{#2\cslentryspacingunit}
 }%
 {}
\usepackage{calc}
\newcommand{\CSLBlock}[1]{#1\hfill\break}
\newcommand{\CSLLeftMargin}[1]{\parbox[t]{\csllabelwidth}{#1}}
\newcommand{\CSLRightInline}[1]{\parbox[t]{\linewidth - \csllabelwidth}{#1}\break}
\newcommand{\CSLIndent}[1]{\hspace{\cslhangindent}#1}
\ifLuaTeX
  \usepackage{selnolig}  % disable illegal ligatures
\fi
\IfFileExists{bookmark.sty}{\usepackage{bookmark}}{\usepackage{hyperref}}
\IfFileExists{xurl.sty}{\usepackage{xurl}}{} % add URL line breaks if available
\urlstyle{same}
\hypersetup{
  pdftitle={Wamu: A Protocol for Building Threshold Signature Wallets Controlled by Multiple Decentralized Identities},
  hidelinks,
  pdfcreator={LaTeX via pandoc}}

\title{Wamu: A Protocol for Building Threshold Signature Wallets
Controlled by Multiple Decentralized Identities}
\author{David Semakula\\
hello@davidsemakula.com\\
https://davidsemakula.com}
\date{16th May, 2023}

\begin{document}
\maketitle

{
\setcounter{tocdepth}{3}
\tableofcontents
}
\hypertarget{introduction}{%
\subsection{1. Introduction}\label{introduction}}

Multisig wallets (e.g.~Safe \footnote{Safe. \url{https://safe.global}})
are already widely adopted \footnote{Dune Analytics. {[}Mainnet{]} Safe.
  \url{https://dune.com/safe/ethereum}} and have proven the importance
of noncustodial shared wallets with threshold access structures
controlled by multiple decentralized identities, for mainstream users
and decentralized teams and organizations.

However, threshold signature wallets have some unique benefits over
multisig wallets including: cost-effectiveness, universal
interoperability, and enhanced privacy and security.

This is because while multiple parties each independently sign a
transaction and the set of signatures is evaluated against the access
structure/security policy on-chain for multisig wallets, threshold
signature wallets instead use a threshold signature scheme to allow
multiple parties to jointly compute a single signature that's similar to
those computed by traditional wallets (e.g.~Metamask \footnote{MetaMask.
  \url{https://metamask.io}}).

\hypertarget{problem}{%
\subsubsection{1.1. Problem}\label{problem}}

Despite the aforementioned benefits, there is currently no mainstream
threshold signature wallet alternative to multisig wallets for
decentralized teams and organizations that require noncustodial shared
wallets with threshold access structures because:

\begin{itemize}
\tightlist
\item
  Most mainstream threshold signature wallets (e.g.~ZenGo \footnote{ZenGo.
    \url{https://zengo.com}} and Torus \footnote{Torus.
    \url{https://tor.us}}) are designed for the single-user setting with
  each party simply being either a separate device or authentication
  factor for the same user.
\item
  Most institutional threshold signature wallet solutions
  (e.g.~Fireblocks \footnote{Fireblocks.
    \url{https://www.fireblocks.com}}, Sepior \footnote{Sepior.
    \url{https://sepior.com}} and Taurus \footnote{Taurus.
    \url{https://www.taurushq.com}}) have architectures that are either
  infeasible and/or undesirable for decentralized teams and
  organizations because of one or more of the following requirements:

  \begin{itemize}
  \tightlist
  \item
    Centralized or trust-based identity infrastructure for
    authenticating signing parties.
  \item
    Controlled network environments with low latency and/or persistent
    synchronous connections between signing parties.
  \end{itemize}
\end{itemize}

\hypertarget{solution}{%
\subsubsection{1.2. Solution}\label{solution}}

The ecosystem needs a new breed of noncustodial threshold signature
wallet solutions that are controlled by multiple decentralized
identities and can run on mainstream consumer devices making them well
suited for use by decentralized teams and organizations, and mainstream
users.

Recent breakthroughs in threshold signing research have yielded
non-interactive threshold signature schemes (e.g.~CGGMP20
{[}\protect\hyperlink{ref-cggmp20}{1}{]}, GG20
{[}\protect\hyperlink{ref-gg20}{2}{]} and CMP20
{[}\protect\hyperlink{ref-cmp20}{3}{]}) that allow for asynchronous
communication between signing parties, making the use of mainstream
consumer devices as signing parties viable.

To remove the need for centralized and/or trust-based identity systems,
and provide a user experience similar to existing multisig wallets, Wamu
introduces a unique approach of augmenting a state-of-the-art
non-interactive threshold signature scheme (e.g.~CGGMP20
{[}\protect\hyperlink{ref-cggmp20}{1}{]}) by cryptographically
associating each signing party with a decentralized identity. This is
achieved by:

\begin{itemize}
\tightlist
\item
  Splitting the secret share for each party between the party and the
  output of a signing operation by its associated decentralized identity
  thus making the signing operation a requirement for reconstructing the
  party's secret share as described in
  \protect\hyperlink{share-splitting-and-reconstruction}{section 3}.
\item
  Adding peer-to-peer decentralized identity verification to the key
  generation and signing protocols (and optionally to the key refresh
  protocol) of the threshold signature scheme.
\item
  Defining protocols for identity rotation, share addition and removal,
  threshold modification and share recovery (as described in
  \protect\hyperlink{share-recovery}{section 4}) that build on top of
  the above 2 augmentations.
\end{itemize}

\textbf{NOTE:} For interoperability with existing wallet solutions, the
only requirement for decentralized identity providers is the ability to
compute cryptographic signatures for any arbitrary message in such a way
that the output signature can be verified in a non-interactive manner.

\hypertarget{preliminaries}{%
\subsection{2. Preliminaries}\label{preliminaries}}

The rest of this document describes how Wamu's unique share splitting
and reconstruction, and share recovery protocols work. For these
descriptions, we'll use the following notation:

\begin{itemize}
\tightlist
\item
  \(P\) denotes a party.
\item
  \(I\) denotes a decentralized identity.
\item
  \(sk\) denotes the secret key of a decentralized identity.
\item
  \(Sig\) denotes a signing algorithm.
\item
  \(q\) denotes the prime order of cyclic group of the elliptic curve.
\end{itemize}

\textbf{NOTE:} While the share splitting and reconstruction protocol is
described in technical detail in this document, for simplicity, the
share recovery protocol is only described at a high-level and no
technical detail is provided for decentralized identity verification and
rest of Wamu's sub-protocols. We refer the reader to Wamu's technical
specification for the technical details that are not provided in this
document.

\hypertarget{share-splitting-and-reconstruction}{%
\subsection{3. Share Splitting and
Reconstruction}\label{share-splitting-and-reconstruction}}

Assuming that we have a secret share \(x\) for a party \(P\) with an
associated decentralized identity \(I\), the share splitting and
reconstruction protocol describes how to split \(x\) between \(P\) and
the output of a signing operation \(Sig\) by \(I\) so that the output of
\(Sig\) is required to reconstruct the secret share \(x\).

This is achieved by generating a message \(m\) (we'll refer to this
message as the ``signing share'') and computing a ``sub-share''
\(\beta\) (i.e a share of the secret share \(x\)) in such a way that
\(m\) needs to be signed by \(I\) using \(Sig\) to produce another
``sub-share'' \(\alpha\), such that \(\alpha\) and \(\beta\) are shares
of \(x\) under Shamir's secret-sharing scheme
{[}\protect\hyperlink{ref-sss79}{4}{]}.

\textbf{NOTE:} Share splitting and reconstruction is a single-party
localized concern that happens after (and is not related to) the
distributed key generation (DKG) protocol of the threshold signature
scheme.

\hypertarget{share-splitting}{%
\subsubsection{3.1. Share splitting}\label{share-splitting}}

Given a secret share \(x\) as input and access to the decentralized
identity \(I\) with secret key \(sk\), the share splitting protocol
proceeds as follows:

\begin{enumerate}
\def\labelenumi{\arabic{enumi}.}
\tightlist
\item
  Sample a random message \(m\) (i.e.~the signing share).
\item
  Compute a signature \((r, s) = Sig(sk, m)\).
\item
  Compute the first sub-share of \(x\) as the point
  \(\alpha = (r, s \bmod q)\).
\item
  Generate a line \(L\) (i.e a polynomial of degree 1) such that
  \(\alpha\) is a point on the line and \(x\) is the constant term
  (i.e.~Polynomial Interpolation
  {[}\protect\hyperlink{ref-wiki:interpolation}{5}{]})
\item
  Compute another point \(\beta\) from \(L\) such that
  \(\beta \neq \alpha\), \(\beta\) becomes the second sub-share of
  \(x\).
\item
  Erase both \(\alpha\) and \(L\) from memory.
\item
  Return the signing share \(m\) and the sub-share \(\beta\).
\end{enumerate}

\hypertarget{share-reconstruction}{%
\subsubsection{3.2. Share reconstruction}\label{share-reconstruction}}

Given a signing share \(m\) and a sub-share \(\beta\) as input (i.e.~the
outputs of the share splitting protocol in
\protect\hyperlink{share-splitting}{section 3.1} above) and access to
the decentralized identity \(I\) with secret key \(sk\), the share
reconstruction protocol proceeds as follows:

\begin{enumerate}
\def\labelenumi{\arabic{enumi}.}
\tightlist
\item
  Compute a signature \((r, s) = Sig(sk, m)\).
\item
  Compute a sub-share \(\alpha\) as the point
  \(\alpha = (r, s \bmod q)\).
\item
  Generate a line \(L\) by performing Polynomial Interpolation
  {[}\protect\hyperlink{ref-wiki:interpolation}{5}{]} using \(\alpha\)
  and \(\beta\) as inputs.
\item
  Compute \(x\) as the constant term of \(L\).
\item
  Erase both \(\alpha\) and \(L\) from memory.
\item
  Return \(x\) as the secret share.
\end{enumerate}

\textbf{NOTE:} For ECDSA signatures, the value of the parameter \(s\) in
\((r, s) = Sig(sk, m)\) is already computed modulo \(q\). We use the
notation \(\alpha = (r, s \bmod q)\) for the sub-share to make it clear
(at a glance) that the sub-shares are computed using finite field
arithmetic.

\hypertarget{share-recovery}{%
\subsection{4. Share Recovery}\label{share-recovery}}

Share recovery is only possible if the user's decentralized identity
either survived or can be recovered after the disastrous event. In
either case, there are two options for share recovery depending on:

\begin{itemize}
\tightlist
\item
  A quorum of honest parties surviving the disastrous event.
\item
  A backup (preferably encrypted) of a signing share \(m\) and sub-share
  \(\beta\) pair on user-controlled secondary or device-independent
  storage.
\end{itemize}

\hypertarget{share-recovery-quorum}{%
\subsubsection{4.1. Share recovery with a surviving quorum of honest
parties}\label{share-recovery-quorum}}

If a quorum of honest parties survives the disastrous event, share
recovery can be accomplished based on peer-to-peer decentralized
identity verification.

The party \(P_i\) that needs to recover its secret share initiates a
signature-authenticated share recovery request leveraging its associated
decentralized identity \(I_i\). The surviving quorum of honest parties
collectively verify the request, and then initiate the key refresh
protocol of the threshold signature scheme with \(P_i\) participating if
\(I_i\) matches a previously verified decentralized identity for a
signatory.

\hypertarget{share-recovery-backup}{%
\subsubsection{4.2. Share recovery with a backup on user-controlled
secondary or device-independent storage}\label{share-recovery-backup}}

\hypertarget{share-recovery-backup-overview}{%
\paragraph{4.2.1. Overview of share recovery with a
backup}\label{share-recovery-backup-overview}}

From the share splitting and reconstruction protocol in
\protect\hyperlink{share-splitting-and-reconstruction}{section 3} above,
we note that for any party \(P\), the combination of a signing share
\(m\) and a sub-share \(\beta\) alone is insufficient to reconstruct the
secret share \(x\). This is because a signature of \(m\) from the
decentralized identity \(I\) is required to compute the sub-share
\(\alpha\), so that \(\alpha\) and \(\beta\) can then be used to
reconstruct \(L\) and compute the secret share \(x\) as the constant
term of \(L\).

Therefore, a signing share \(m\) and sub-share \(\beta\) pair can be
safely backed up to user-controlled secondary (e.g.~a secondary device
or a flash drive) or device-independent storage (e.g.~Apple iCloud
\footnote{Apple iCloud. \url{https://www.icloud.com}.}, Google Drive
\footnote{Google Drive. \url{https://drive.google.com}.}, Microsoft
OneDrive \footnote{Microsoft OneDrive.
  \url{https://www.microsoft.com/en-us/microsoft-365/onedrive/online-cloud-storage}.},
Dropbox \footnote{Dropbox. \url{https://www.dropbox.com}.} e.t.c)
without exposing the secret share \(x\).

\hypertarget{share-recovery-backup-encrypted}{%
\paragraph{4.2.2. Share recovery with an encrypted
backup}\label{share-recovery-backup-encrypted}}

For increased security, a signature of a standardized phrase can be used
as entropy for generating an encryption secret which can then be used to
encrypt the signing share \(m\) and the sub-share \(\beta\) using a
symmetric encryption algorithm before saving them to back up storage.
Share recovery would then start by signing this standardized phrase,
using the signature to recreate the encryption secret and then
decrypting the encrypted backup to retrieve the signing share \(m\) and
the sub-share \(\beta\).

\hypertarget{share-recovery-backup-enhancements}{%
\paragraph{4.2.3. Further security and usability considerations for
share recovery with a backup}\label{share-recovery-backup-enhancements}}

For further improved security and usability, the signing share \(m\) can
be prefixed with a custom message that alerts the user to the purpose of
the signature. This can help reduce the effectiveness of an adversary
that gains access to the backup and tries to trick the user into signing
\(m\).

Additionally, it's possible to rerun the share splitting protocol to
generate a new pair of a signing share \(m^ \ast\) and a sub-share
\(\beta ^ \ast\) such that \(m^ \ast \neq m\),
\(\beta ^ \ast \neq \beta\) and \(L^ \ast \neq L\) to be specifically
used for backup and recovery. This gives us the option to have separate
signing shares for backup and recovery with customized prefixes that
make it clear to the user that they're signing a backup signing share.

Lastly, the ``backup'' signing share \(m^ \ast\) can be generated based
on user input (e.g.~a passphrase or security questions) removing the
need for it to be backed up together with a sub-share \(\beta ^ \ast\)
but instead relying on the user to provide this input during recovery as
a security-usability tradeoff.

\hypertarget{conclusion}{%
\subsection{5. Conclusion}\label{conclusion}}

The Wamu project (meaning ``together'') aims to unlock the benefits of
threshold signature wallets for decentralized teams and organizations,
and mainstream users that require noncustodial shared wallets with
threshold access structures by:

\begin{itemize}
\tightlist
\item
  Defining an open protocol that encourages research into and
  development of mainstream multi-user threshold signature wallet
  solutions.
\item
  Providing modular and performant, free and open-source building blocks
  that allow software developers to either build new mainstream
  multi-user threshold signature wallets or integrate state-of-the-art
  threshold signature schemes into existing mainstream wallets.
\end{itemize}

\hypertarget{references}{%
\subsection{6. References}\label{references}}

\hypertarget{refs}{}
\begin{CSLReferences}{0}{0}
\leavevmode\vadjust pre{\hypertarget{ref-cggmp20}{}}%
\CSLLeftMargin{{[}1{]} }%
\CSLRightInline{Canetti, R., Gennaro, R., Goldfeder, S., Makriyannis, N.
and Peled, U. 2020. UC non-interactive, proactive, threshold ECDSA with
identifiable aborts. \emph{Proceedings of the 2020 ACM SIGSAC conference
on computer and communications security} (New York, NY, USA, 2020),
1769--1787. \url{https://eprint.iacr.org/2021/060}.}

\leavevmode\vadjust pre{\hypertarget{ref-gg20}{}}%
\CSLLeftMargin{{[}2{]} }%
\CSLRightInline{Gennaro, R. and Goldfeder, S. 2020. One round threshold
ECDSA with identifiable abort. Cryptology ePrint Archive, Paper
2020/540. \url{https://eprint.iacr.org/2020/540}.}

\leavevmode\vadjust pre{\hypertarget{ref-cmp20}{}}%
\CSLLeftMargin{{[}3{]} }%
\CSLRightInline{Canetti, R., Makriyannis, N. and Peled, U. 2020. UC
non-interactive, proactive, threshold ECDSA. Cryptology ePrint Archive,
Paper 2020/492. \url{https://eprint.iacr.org/2020/492}.}

\leavevmode\vadjust pre{\hypertarget{ref-sss79}{}}%
\CSLLeftMargin{{[}4{]} }%
\CSLRightInline{Shamir, A. 1979. How to share a secret. \emph{Commun.
ACM}. 22, 11 (Nov. 1979), 612--613.
DOI:https://doi.org/\href{https://doi.org/10.1145/359168.359176}{10.1145/359168.359176}.}

\leavevmode\vadjust pre{\hypertarget{ref-wiki:interpolation}{}}%
\CSLLeftMargin{{[}5{]} }%
\CSLRightInline{Wikipedia. Polynomial interpolation:
\href{https://en.wikipedia.org/wiki/Polynomial_interpolation}{\emph{https://en.wikipedia.org/wiki/Polynomial\_interpolation}}.
Accessed: 2023-05-12.}

\end{CSLReferences}

\end{document}
