% Options for packages loaded elsewhere
\PassOptionsToPackage{unicode}{hyperref}
\PassOptionsToPackage{hyphens}{url}
%
\documentclass[
]{article}
\usepackage{amsmath,amssymb}
\usepackage{iftex}
\ifPDFTeX
  \usepackage[T1]{fontenc}
  \usepackage[utf8]{inputenc}
  \usepackage{textcomp} % provide euro and other symbols
\else % if luatex or xetex
  \usepackage{unicode-math} % this also loads fontspec
  \defaultfontfeatures{Scale=MatchLowercase}
  \defaultfontfeatures[\rmfamily]{Ligatures=TeX,Scale=1}
\fi
\usepackage{lmodern}
\ifPDFTeX\else
  % xetex/luatex font selection
\fi
% Use upquote if available, for straight quotes in verbatim environments
\IfFileExists{upquote.sty}{\usepackage{upquote}}{}
\IfFileExists{microtype.sty}{% use microtype if available
  \usepackage[]{microtype}
  \UseMicrotypeSet[protrusion]{basicmath} % disable protrusion for tt fonts
}{}
\makeatletter
\@ifundefined{KOMAClassName}{% if non-KOMA class
  \IfFileExists{parskip.sty}{%
    \usepackage{parskip}
  }{% else
    \setlength{\parindent}{0pt}
    \setlength{\parskip}{6pt plus 2pt minus 1pt}}
}{% if KOMA class
  \KOMAoptions{parskip=half}}
\makeatother
\usepackage{xcolor}
\setlength{\emergencystretch}{3em} % prevent overfull lines
\providecommand{\tightlist}{%
  \setlength{\itemsep}{0pt}\setlength{\parskip}{0pt}}
\setcounter{secnumdepth}{-\maxdimen} % remove section numbering
\newlength{\cslhangindent}
\setlength{\cslhangindent}{1.5em}
\newlength{\csllabelwidth}
\setlength{\csllabelwidth}{3em}
\newlength{\cslentryspacingunit} % times entry-spacing
\setlength{\cslentryspacingunit}{\parskip}
\newenvironment{CSLReferences}[2] % #1 hanging-ident, #2 entry spacing
 {% don't indent paragraphs
  \setlength{\parindent}{0pt}
  % turn on hanging indent if param 1 is 1
  \ifodd #1
  \let\oldpar\par
  \def\par{\hangindent=\cslhangindent\oldpar}
  \fi
  % set entry spacing
  \setlength{\parskip}{#2\cslentryspacingunit}
 }%
 {}
\usepackage{calc}
\newcommand{\CSLBlock}[1]{#1\hfill\break}
\newcommand{\CSLLeftMargin}[1]{\parbox[t]{\csllabelwidth}{#1}}
\newcommand{\CSLRightInline}[1]{\parbox[t]{\linewidth - \csllabelwidth}{#1}\break}
\newcommand{\CSLIndent}[1]{\hspace{\cslhangindent}#1}
\ifLuaTeX
  \usepackage{selnolig}  % disable illegal ligatures
\fi
\IfFileExists{bookmark.sty}{\usepackage{bookmark}}{\usepackage{hyperref}}
\IfFileExists{xurl.sty}{\usepackage{xurl}}{} % add URL line breaks if available
\urlstyle{same}
\hypersetup{
  pdftitle={Wamu: A Protocol for Building Threshold Signature Wallets Controlled by Multiple Decentralized Identities},
  hidelinks,
  pdfcreator={LaTeX via pandoc}}

\title{Wamu: A Protocol for Building Threshold Signature Wallets
Controlled by Multiple Decentralized Identities}
\usepackage{etoolbox}
\makeatletter
\providecommand{\subtitle}[1]{% add subtitle to \maketitle
  \apptocmd{\@title}{\par {\large #1 \par}}{}{}
}
\makeatother
\subtitle{Technical Specification}
\author{David Semakula\\
hello@davidsemakula.com\\
https://davidsemakula.com}
\date{20th July, 2023}

\begin{document}
\maketitle

{
\setcounter{tocdepth}{3}
\tableofcontents
}
\hypertarget{introduction}{%
\subsection{1. Introduction}\label{introduction}}

This document describes the Wamu protocol which augments a
state-of-the-art non-interactive threshold signature scheme
(e.g.~CGGMP20 {[}\protect\hyperlink{ref-cggmp20}{1}{]}) by
cryptographically associating each signing party with a decentralized
identity. This is achieved by:

\begin{itemize}
\tightlist
\item
  Splitting the secret share for each party between the party and the
  output of a signing operation by its associated decentralized identity
  thus making the signing operation a requirement for reconstructing the
  party's secret share.
\item
  Adding peer-to-peer decentralized identity verification to the key
  generation and signing protocols (and optionally to the key refresh
  protocol) of the threshold signature scheme.
\item
  Defining protocols for identity rotation, share addition and removal,
  threshold modification and share recovery that build on top of the
  above 2 augmentations.
\end{itemize}

Wamu is designed to operate in a decentralized, trust-minimized and
asynchronous setting with:

\begin{itemize}
\tightlist
\item
  no centralized or trust-based identity infrastructure.
\item
  signing parties being mainstream consumer devices communicating
  asynchronously.
\end{itemize}

\textbf{NOTE:} For interoperability with existing wallet solutions, the
only requirement for decentralized identity providers is the ability to
compute cryptographic signatures for any arbitrary message in such a way
that the output signature can be verified in a non-interactive manner.

\hypertarget{preliminaries}{%
\subsection{2. Preliminaries}\label{preliminaries}}

The rest of this document describes the Wamu protocol in technical
detail. For these descriptions, we'll use the following notation:

\begin{itemize}
\tightlist
\item
  \(P\) denotes a party.
\item
  \(I\) denotes a decentralized identity.
\item
  \(pk\) denotes the address (or public key) of a decentralized
  identity.
\item
  \(sk\) denotes the secret key of a decentralized identity.
\item
  \(KeyGen\) denotes a key generation algorithm.
\item
  \(Sig\) denotes a signing algorithm.
\item
  \(Ver\) denotes a signature verification algorithm.
\item
  \(S\) denotes the set of verified decentralized identities for all
  parties.
\item
  \(q\) denotes the prime order of the cyclic group of the elliptic
  curve.
\item
  \(A\) denotes a predefined prefix chosen to ensure that signatures
  computed for identity authentication cannot be valid transaction
  signatures.
\end{itemize}

\textbf{NOTE:} While the augmenting protocols in this document are
described in relation to the current (circa. 2023) state-of-the-art
CGGMP20 {[}\protect\hyperlink{ref-cggmp20}{1}{]} non-interactive
threshold signature scheme for ECDSA signatures, Wamu is a generic
protocol that can be adapted to any non-interactive threshold signature
scheme (e.g.~GG20 {[}\protect\hyperlink{ref-gg20}{2}{]} and CMP20
{[}\protect\hyperlink{ref-cmp20}{3}{]}) that allows for asynchronous
communication between signing parties.

\hypertarget{share-splitting-and-reconstruction}{%
\subsection{3. Share Splitting and
Reconstruction}\label{share-splitting-and-reconstruction}}

Given a secret share \(x\) for a party \(P\) with an associated
decentralized identity \(I\), the share splitting and reconstruction
protocol describes how to split \(x\) between \(P\) and the output of a
signing operation \(Sig\) by \(I\) so that the output of \(Sig\) is
required to reconstruct the secret share \(x\).

This is achieved by generating a message \(m\) (we'll refer to this
message as the ``signing share'') and computing a ``sub-share''
\(\beta\) (i.e a share of the secret share \(x\)) in such a way that
\(m\) needs to be signed by \(I\) using \(Sig\) to produce another
``sub-share'' \(\alpha\), such that \(\alpha\) and \(\beta\) are shares
of \(x\) under Shamir's secret-sharing scheme
{[}\protect\hyperlink{ref-sss79}{4}{]}.

\textbf{NOTE:} Share splitting and reconstruction is a single-party
localized concern that happens after (and is not related to) the
distributed key generation (DKG) protocol of the threshold signature
scheme.

\hypertarget{share-splitting}{%
\subsubsection{3.1. Share splitting}\label{share-splitting}}

Given a secret share \(x\) as input and access to the decentralized
identity \(I\) with secret key \(sk\), the share splitting protocol
proceeds as follows:

\begin{enumerate}
\def\labelenumi{\arabic{enumi}.}
\tightlist
\item
  Sample a random message \(m\) (i.e.~the signing share).
\item
  Compute a signature \((r, s) = Sig(sk, m)\).
\item
  Compute the first sub-share of \(x\) as the point
  \(\alpha = (r, s) \: (\bmod \, q)\).
\item
  Generate a line \(L\) (i.e a polynomial of degree 1) such that
  \(\alpha\) is a point on the line and \(x\) is the constant term
  (i.e.~Polynomial Interpolation
  {[}\protect\hyperlink{ref-wiki:interpolation}{5}{]})
\item
  Compute another point \(\beta\) from \(L\) such that
  \(\beta \neq \alpha\), \(\beta\) becomes the second sub-share of
  \(x\).
\item
  Erase both \(\alpha\) and \(L\) from memory.
\item
  Return the signing share \(m\) and the sub-share \(\beta\).
\end{enumerate}

\hypertarget{share-reconstruction}{%
\subsubsection{3.2. Share reconstruction}\label{share-reconstruction}}

Given a signing share \(m\) and a sub-share \(\beta\) as input (i.e.~the
outputs of the share splitting protocol in
\protect\hyperlink{share-splitting}{section 3.1}) and access to the
decentralized identity \(I\) with secret key \(sk\), the share
reconstruction protocol proceeds as follows:

\begin{enumerate}
\def\labelenumi{\arabic{enumi}.}
\tightlist
\item
  Compute a signature \((r, s) = Sig(sk, m)\).
\item
  Compute a sub-share \(\alpha\) as the point
  \(\alpha = (r, s) \: (\bmod \, q)\).
\item
  Generate the line \(L\) by performing Polynomial Interpolation
  {[}\protect\hyperlink{ref-wiki:interpolation}{5}{]} using \(\alpha\)
  and \(\beta\) as inputs.
\item
  Compute \(x\) as the constant term of \(L\).
\item
  Erase both \(\alpha\) and \(L\) from memory.
\item
  Return \(x\) as the secret share.
\end{enumerate}

\textbf{NOTE:} For ECDSA signatures, the parameters \(r\) and \(s\) in
\((r, s) = Sig(sk, m)\) are already computed modulo \(q\). We use the
notation \(\alpha = (r, s) \: (\bmod \, q)\) for the sub-share to make
it clear (at a glance) that the sub-shares are computed using finite
field arithmetic.

\hypertarget{key-generation}{%
\subsection{4. Key Generation}\label{key-generation}}

Follow the key generation protocol described in section 3.1 and figure 5
of CGGMP20 {[}\protect\hyperlink{ref-cggmp20}{1}{]} to generate ECDSA
secret shares with the following modifications:

\begin{enumerate}
\def\labelenumi{\arabic{enumi}.}
\tightlist
\item
  At the end of Round 1, broadcast 2 additional parameters for each
  \(P_i\) associated with the decentralized identity \(I_i\) with
  address \(pk_i\) and secret key \(sk_i\) as follows:

  \begin{itemize}
  \tightlist
  \item
    The decentralized identity address \(pk_i\).
  \item
    The signature \(\varphi _i = Sig(sk_i, A | V_i)\).
  \end{itemize}
\item
  At the beginning of Round 2, for each \(P_i\), verify \(\varphi _j\)
  from all \(P_j\) where \(j \neq i\) by checking that the output of
  \(Ver(pk_j, A | V_j, \varphi _j)\) is valid or report the culprit and
  halt.
\item
  After the Output phase, follow the share splitting protocol in
  \protect\hyperlink{share-splitting}{section 3.1} to split secret share
  \(x_i\) into a signing share \(m_i\) and a sub-share \(\beta _i\) for
  each party \(P_i\).
\item
  Modify Stored State for each \(P_i\) as follows:

  \begin{itemize}
  \tightlist
  \item
    Don't store \(x_i\).
  \item
    Add \(pk_i\), \(m_i\), \(\beta _i\) and
    \(S_i = \{ pk_j : i \neq j \}\) (i.e the set of verified
    decentralized identities for all other parties).
  \end{itemize}
\end{enumerate}

\hypertarget{key-refresh}{%
\subsection{5. Key Refresh}\label{key-refresh}}

Follow the key refresh protocol described in section 3.2 and figure 6 of
CGGMP20 {[}\protect\hyperlink{ref-cggmp20}{1}{]} to generate new ECDSA
secret shares with the following modifications:

\begin{enumerate}
\def\labelenumi{\arabic{enumi}.}
\tightlist
\item
  At the end of Round 1, broadcast 2 additional parameters for each
  \(P_i\) associated with the decentralized identity \(I_i\) with
  address \(pk_i\) and secret key \(sk_i\) as follows:

  \begin{itemize}
  \tightlist
  \item
    The decentralized identity address \(pk_i\).
  \item
    The signature \(\varphi _i = Sig(sk_i, A | V_i)\).
  \end{itemize}
\item
  At the beginning of Round 2, for each \(P_i\), verify \(\varphi _j\)
  from all \(P_j\) where \(j \neq i\) as follows:

  \begin{itemize}
  \tightlist
  \item
    Verify that \(pk_i \in S_j\) or report the culprit and halt.
  \item
    Verify \(\varphi _i\) by checking that the output of
    \(Ver(pk_j, A | V_j, \varphi _j)\) is valid or report the culprit
    and halt.
  \end{itemize}
\item
  After the Output phase, follow the share splitting protocol in
  \protect\hyperlink{share-splitting}{section 3.1} to split the new
  secret share \(x_i^\ast\) into a new signing share \(m_i^\ast\) and a
  new sub-share \(\beta _i^\ast\) for each party \(P_i\).
\item
  Modify Stored State for each \(P_i\) as follows:

  \begin{itemize}
  \tightlist
  \item
    Don't store \(x_i^\ast\).
  \item
    Replace \(m_i\) with \(m_i^\ast\) and \(\beta _i\) with
    \(\beta _i^\ast\).
  \end{itemize}
\end{enumerate}

\hypertarget{signing}{%
\subsection{6. Signing}\label{signing}}

Follow the signing protocol described in sections 4.2 and 4.3 and figure
8 of CGGMP20 {[}\protect\hyperlink{ref-cggmp20}{1}{]} to generate an
ECDSA signature with the following modifications:

\begin{enumerate}
\def\labelenumi{\arabic{enumi}.}
\tightlist
\item
  Before Round 1, for each party \(P_i\), follow the share
  reconstruction protocol in
  \protect\hyperlink{share-reconstruction}{section 3.2} to reconstruct
  secret share \(x_i\).
\item
  At the end of Round 1, for each \(P_i\) associated with the
  decentralized identity \(I_i\) with address \(pk_i\) and secret key
  \(sk_i\), send 2 additional parameters to all \(P_j\) where
  \(j \neq i\) as follows:

  \begin{itemize}
  \tightlist
  \item
    The decentralized identity address \(pk_i\).
  \item
    The signature \(\varphi _i = Sig(sk_i, A | m)\).
  \end{itemize}
\item
  At the beginning of the Output phase, verify \(\varphi _j\) from all
  \(P_j\) where \(j \neq i\) as follows:

  \begin{itemize}
  \tightlist
  \item
    Verify that \(pk_i \in S_j\) or report the culprit and halt.
  \item
    Verify \(\varphi _i\) by checking that the output of
    \(Ver(pk_i, A | m, \varphi _i)\) is valid or report the culprit and
    halt.
  \end{itemize}
\end{enumerate}

\hypertarget{identity-authed-request}{%
\subsection{7. Identity Authenticated Request Initiation and
Verification}\label{identity-authed-request}}

Decentralized identity authenticated requests allow parties to perform
or request actions based on their associated decentralized identity.

\hypertarget{identity-authed-request-initiation}{%
\subsubsection{7.1. Identity Authenticated Request
Initiation}\label{identity-authed-request-initiation}}

To initiate an identity authenticated request with a command \(C\) from
a party \(P_i\) associated with decentralized identity \(I_i\) with
address \(pk_i\) and secret key \(sk_i\):

\begin{enumerate}
\def\labelenumi{\arabic{enumi}.}
\tightlist
\item
  Read the current UTC timestamp \(t\).
\item
  Compute the signature \(\varphi = Sig(sk_i, A | C | t)\).
\item
  Broadcast \(C\), \(pk_i\), \(t\) and \(\varphi\).
\end{enumerate}

\hypertarget{identity-authed-request-verification}{%
\subsubsection{7.2. Identity Authenticated Request
Verification}\label{identity-authed-request-verification}}

To verify an identity authenticated request with a command \(C\) from a
party \(P_i\) given its associated decentralized identity address
\(pk_i\), a timestamp \(t\), a signature \(\varphi\) and a set of
verified decentralized identities for all other parties \(S_j\) as
input:

\begin{enumerate}
\def\labelenumi{\arabic{enumi}.}
\tightlist
\item
  Verify that \(pk_i \in S_j\) or report the culprit and halt.
\item
  Verify that \(t\) is within the current epoch for identity
  authenticated requests or report the culprit and halt.
\item
  Verify \(\varphi\) by checking that the output of
  \(Ver(pk_i, A | C | t, \varphi)\) is valid or report the culprit and
  halt.
\end{enumerate}

\hypertarget{identity-challenge}{%
\subsection{8. Identity Challenge}\label{identity-challenge}}

Identity challenges are used to verify that a party controls a
decentralized identity.

\hypertarget{identity-challenge-initiation}{%
\paragraph{8.1. Identity Challenge
Initiation}\label{identity-challenge-initiation}}

To issue an identity challenge to a party \(P_i\) from all verifying
parties \(P_j\) where \(j \neq i\): 1. Sample a random \(v_j\). 2.
Broadcast \(v_j\) to all parties, such that all parties can compute
\(v = | _{j \neq i} \: v_j\).

\hypertarget{identity-challenge-response}{%
\paragraph{8.2. Identity Challenge
Response}\label{identity-challenge-response}}

For a party \(P_i\) with associated decentralized identity secret key
\(sk_i\), to respond to an identity challenge given \(v_j\) from all
parties \(P_j\) where \(j \neq i\):

\begin{enumerate}
\def\labelenumi{\arabic{enumi}.}
\tightlist
\item
  Compute \(v = | _{j \neq i} \: v_j\).
\item
  Compute the signature \(\psi = Sig(sk_i, A | v)\).
\item
  Broadcast \(\psi\) to all verifying parties \(P_j\).
\end{enumerate}

\hypertarget{identity-challenge-verification}{%
\paragraph{8.3. Identity Challenge Response
Verification}\label{identity-challenge-verification}}

To verify an identity challenge response from a party \(P_i\) given its
associated decentralized identity address \(pk_i\), a signature \(\psi\)
and \(v_j\) from all verifying parties \(P_j\) where \(j \neq i\) as
input:

\begin{enumerate}
\def\labelenumi{\arabic{enumi}.}
\tightlist
\item
  Compute \(v = | _{j \neq i} \: v_j\).
\item
  Verify \(\psi\) by checking that the output of
  \(Ver(pk_i, A | v, \psi)\) is valid or report the culprit and halt.
\end{enumerate}

\hypertarget{identity-rotation}{%
\subsection{9. Identity Rotation}\label{identity-rotation}}

Identity rotation allows any party to change the decentralized identity
associated with its secret share.

Identity rotation for a party \(P_i\) from a decentralized identity
\(I_i\) with address \(pk_i\) and secret key \(sk_i\) to a decentralized
identity \(I_i^ \ast\) with address \(pk_i^ \ast\) and secret key
\(sk_i^ \ast\) proceeds as follows:

\begin{enumerate}
\def\labelenumi{\arabic{enumi}.}
\tightlist
\item
  For \(P_i\), initiate an ``identity-rotation'' request by following
  the protocol in
  \protect\hyperlink{identity-authed-request-initiation}{section 7.1}.
\item
  For all \(P_j\) where \(j \neq i\):

  \begin{itemize}
  \tightlist
  \item
    Verify the ``identity-rotation'' request by following the protocol
    in \protect\hyperlink{identity-authed-request-verification}{section
    7.2}.
  \item
    Initiate an identity challenge for \(P_i\) by following the protocol
    in \protect\hyperlink{identity-challenge-initiation}{section 8.1}.
  \end{itemize}
\item
  For \(P_i\), respond to the identity challenge by following the
  protocol in \protect\hyperlink{identity-challenge-response}{section
  8.2} with the following augmentations:

  \begin{itemize}
  \tightlist
  \item
    Compute an additional signature
    \(\psi _i^ \ast = Sig(sk_i^ \ast, A | v)\).
  \item
    Add \(pk_i^ \ast\) and \(\psi _i^ \ast\) to the broadcast
    parameters.
  \end{itemize}
\item
  For all \(P_j\) where \(j \neq i\):

  \begin{itemize}
  \tightlist
  \item
    Verify the identity challenge response from \(P_i\) by following the
    protocol in
    \protect\hyperlink{identity-challenge-verification}{section 8.3}.
  \item
    Verify that \(P_i\) controls the new decentralized identity address
    \(pk_i^ \ast\) as follows:

    \begin{itemize}
    \tightlist
    \item
      Compute \(v = | _{j \neq i} \: v_j\):
    \item
      Verify \(\psi ^ \ast\) by checking that the output of
      \(Ver(pk_i^ \ast, A | v, \psi ^ \ast)\) is valid or report the
      culprit and halt.
    \end{itemize}
  \item
    Modify Stored State as follows:

    \begin{itemize}
    \tightlist
    \item
      Create \(S_i^ \ast\) by replacing \(pk_i\) with \(pk_i^ \ast\) in
      \(S_i\).
    \item
      Replace \(S_i\) with \(S_i^ \ast\).
    \end{itemize}
  \item
    Send confirmation of successful rotation of the identity to \(P_i\).
  \end{itemize}
\item
  For \(P_i\), upon receiving confirmation of successful rotation from a
  quorum of \(P_j\):

  \begin{itemize}
  \tightlist
  \item
    Compute the new signing share \(m_i^ \ast\) and sub-share
    \(\beta _i^ \ast\) based on the new decentralized identity
    \(I_i^ \ast\) as follows:

    \begin{itemize}
    \tightlist
    \item
      Compute the secret share \(x_i\) by following the share
      reconstruction protocol in
      \protect\hyperlink{share-reconstruction}{section 3.2}.
    \item
      Follow the share splitting protocol in
      \protect\hyperlink{share-splitting}{section 3.1} to split \(x_i\)
      into a new signing share \(m_i^ \ast\) and a new sub-share
      \(\beta _i^ \ast\) based on the new decentralized identity
      \(I_i^ \ast\).
    \end{itemize}
  \item
    Modify Stored State as follows:

    \begin{itemize}
    \tightlist
    \item
      Replace \(pk_i\) with \(pk_i^ \ast\).
    \item
      Replace \(m_i\) with \(m_i^ \ast\).
    \item
      Replace \(\beta _i\) with \(\beta _i^ \ast\).
    \end{itemize}
  \end{itemize}
\end{enumerate}

\hypertarget{quorum-approved-request}{%
\subsection{10. Quorum Approved Request Initiation and
Verification}\label{quorum-approved-request}}

Quorum approved requests allow any verified party to initiate actions
that require explicit approval from a quorum of verified parties before
execution (e.g.~share addition and removal, and threshold modification).

A quorum approved request with a command \(C\) from a party \(P_i\)
associated with decentralized identity \(I_i\) with address \(pk_i\) and
secret key \(sk_i\) proceeds as follows:

\begin{enumerate}
\def\labelenumi{\arabic{enumi}.}
\tightlist
\item
  For \(P_i\), initiate an identity authenticated request by following
  the protocol in
  \protect\hyperlink{identity-authed-request-initiation}{section 7.1}.
\item
  For all \(P_j\) where \(j \neq i\) that approve the requested action:

  \begin{itemize}
  \tightlist
  \item
    Verify the identity authenticated request by following the protocol
    in \protect\hyperlink{identity-authed-request-verification}{section
    7.2}.
  \item
    Initiate an identity challenge for \(P_i\) by following the protocol
    in \protect\hyperlink{identity-challenge-initiation}{section 8.1}
    with the following augmentations:

    \begin{itemize}
    \tightlist
    \item
      Compute a signature \(\phi _j = Sig(sk_j, A | v_j | C | t)\).
    \item
      Add \(pk_j\) and \(\phi _j\) to the broadcast parameters.
    \end{itemize}
  \end{itemize}
\item
  For \(P_i\), upon receiving an augmented identity challenge from a
  quorum \(S_c\) such that \(S_c \subset S_j\) and \(j \neq i\), respond
  to the identity challenge by following the protocol in
  \protect\hyperlink{identity-challenge-response}{section 8.2} with the
  following modifications:

  \begin{itemize}
  \tightlist
  \item
    At the beginning of the identity challenge response protocol, verify
    that approvals have been received from a quorum \(S_c\) by checking
    that \(\exists \, S_c \subset S_j\) such that
    \(\forall \, pk_c \in S_c\) where \(c \neq i\), the output of
    \(Ver(pk_c, A | v_c | C | t, \phi _c)\) is valid or halt.
  \item
    Compute \(v\) as \(v = | _{c \neq i} \: v_c\).
  \item
    Add \(S_c\) to the broadcast parameters.
  \end{itemize}
\item
  For all \(P_j\) where \(j \neq i\):

  \begin{itemize}
  \tightlist
  \item
    Verify the augmented identity challenge response from \(P_i\) by
    following the protocol in
    \protect\hyperlink{identity-challenge-verification}{section 8.3}
    with the following modifications:

    \begin{itemize}
    \tightlist
    \item
      Compute \(v\) as \(v = | _{c \neq i} \: v_c\).
    \end{itemize}
  \item
    Verify that a valid quorum \(S_c\) such that \(S_c \subset S_j\) and
    \(j \neq i\) has approved the request as follows:

    \begin{itemize}
    \tightlist
    \item
      Verify that \(\forall \, pk_c \in S_c , \: pk_c \in S_j\) where
      \(j \neq i\) and \(c \neq i\) or report the culprit and halt:
    \item
      Verify that \(\forall \, pk_c \in S_c\) where \(c \neq i\), the
      output of \(Ver(pk_c, A | v_c | C | t, \phi _c)\) is valid or
      report the culprit and halt.
    \end{itemize}
  \end{itemize}
\end{enumerate}

\hypertarget{share-addition-and-removal}{%
\subsection{11. Share Addition and
Removal}\label{share-addition-and-removal}}

Share addition and removal allows a quorum of verified parties to either
issue a secret share to a new party and its associated decentralized
identity, or revoke the secret share of any party respectively.

\hypertarget{share-addition}{%
\subsubsection{11.1. Share Addition}\label{share-addition}}

Share addition for a new party \(P_i\) with associated decentralized
identity \(I_i\) proceeds as follows:

\begin{enumerate}
\def\labelenumi{\arabic{enumi}.}
\tightlist
\item
  Initiate a quorum approved ``share-addition'' request by following the
  protocol in \protect\hyperlink{quorum-approved-request}{section 10}.
\item
  Follow the key refresh protocol described in
  \protect\hyperlink{key-refresh}{section 5} with \(P_i\) included as a
  participant if the quorum approved request above succeeded.
\end{enumerate}

\hypertarget{share-removal}{%
\subsubsection{11.2. Share Removal}\label{share-removal}}

Share removal for a party \(P_i\) with associated decentralized identity
\(I_i\) proceeds as follows:

\begin{enumerate}
\def\labelenumi{\arabic{enumi}.}
\tightlist
\item
  Initiate a quorum approved ``share-removal'' request by following the
  protocol in \protect\hyperlink{quorum-approved-request}{section 10}.
\item
  Follow the key refresh protocol described in
  \protect\hyperlink{key-refresh}{section 5} without \(P_i\) if the
  quorum approved request above succeeded.
\end{enumerate}

\hypertarget{threshold-modification}{%
\subsection{12. Threshold Modification}\label{threshold-modification}}

Threshold modification allows a quorum of verified parties to change the
threshold (i.e.~change the size of the quorum).

While threshold modification (or more generally \(t\)-out-of-\(n\)
sharing, and specifically the case where \(n > t+1\)) is not formally
specified in CGGMP20 {[}\protect\hyperlink{ref-cggmp20}{1}{]}, it can be
derived in a relatively straightforward manner based on GG18
{[}\protect\hyperlink{ref-gg18}{6}{]} (and GG20
{[}\protect\hyperlink{ref-gg20}{2}{]}) which CGGMP20
{[}\protect\hyperlink{ref-cggmp20}{1}{]} builds upon (see sections
1.2.8, 1.2.1 and 1.2.2 of CGGMP20
{[}\protect\hyperlink{ref-cggmp20}{1}{]}). In general, CGGMP20
{[}\protect\hyperlink{ref-cggmp20}{1}{]} can be seen as a combination of
CMP20 {[}\protect\hyperlink{ref-cmp20}{3}{]} and GG20
{[}\protect\hyperlink{ref-gg20}{2}{]}, and a direct improvement on GG18
{[}\protect\hyperlink{ref-gg18}{6}{]}.

Therefore, threshold modification can be achieved by following the key
refresh protocol described in section 3.2 and figure 6 of CGGMP20
{[}\protect\hyperlink{ref-cggmp20}{1}{]} and
\protect\hyperlink{key-refresh}{section 5} of this document, with some
modifications based on the key generation protocols described in GG18
{[}\protect\hyperlink{ref-gg18}{6}{]} and GG20
{[}\protect\hyperlink{ref-gg20}{2}{]}, and following the instructions in
section 1.2.8 of CGGMP20 {[}\protect\hyperlink{ref-cggmp20}{1}{]}.

In particular, this entails performing a \(t\)-out-of-\(n\) Feldman's
VSS {[}\protect\hyperlink{ref-feldman-vss}{7}{]} sharing of the values
\(x_i^k\) (as defined in section 3.2 of CGGMP20
{[}\protect\hyperlink{ref-cggmp20}{1}{]}), with the new threshold \(t\)
used as the threshold parameter (similarly defined as \(t\)) for
Feldman's VSS {[}\protect\hyperlink{ref-feldman-vss}{7}{]} protocol as
described in section 2.8 and phase 2 of section 3.1 in GG20
{[}\protect\hyperlink{ref-gg20}{2}{]} (and similarly in section 2.6 and
phase 2 of section 4.1 in GG18 {[}\protect\hyperlink{ref-gg18}{6}{]}).

Threshold modification would then proceed as follows:

\begin{enumerate}
\def\labelenumi{\arabic{enumi}.}
\tightlist
\item
  Initiate a quorum approved ``threshold-modification'' request by
  following the protocol in
  \protect\hyperlink{quorum-approved-request}{section 10}.
\item
  Follow the key refresh protocol described in
  \protect\hyperlink{key-refresh}{section 5} with the modifications
  described above if the quorum approved request succeeds.
\end{enumerate}

\textbf{NOTE:} Similar modifications can be applied to the signing
protocol described in section 3.1 and figure 5 of CGGMP20
{[}\protect\hyperlink{ref-cggmp20}{1}{]} and
\protect\hyperlink{signing}{section 6} of this document to achieve a
\(t\)-out-of-\(n\) sharing of the secret key for \(n \geq t+1\). In
particular, this entails performing a \(t\)-out-of-\(n\) Feldman's VSS
{[}\protect\hyperlink{ref-feldman-vss}{7}{]} sharing of the value
\(x_i\) (as defined in section 3.1 of CGGMP20
{[}\protect\hyperlink{ref-cggmp20}{1}{]}), based on the same
modifications from GG20 {[}\protect\hyperlink{ref-gg20}{2}{]} and GG18
{[}\protect\hyperlink{ref-gg18}{6}{]} described above, and following the
instructions in section 1.2.8 of CGGMP20
{[}\protect\hyperlink{ref-cggmp20}{1}{]}.

\hypertarget{share-recovery}{%
\subsection{13. Share Recovery}\label{share-recovery}}

Share recovery is only possible if the user's decentralized identity
either survived or can be recovered after the disastrous event. In
either case, there are two options for share recovery depending on:

\begin{itemize}
\tightlist
\item
  A quorum of honest parties surviving the disastrous event.
\item
  A backup (preferably encrypted) of a signing share \(m\) and sub-share
  \(\beta\) pair on user-controlled secondary or device-independent
  storage.
\end{itemize}

\hypertarget{share-recovery-quorum}{%
\subsubsection{13.1. Share recovery with a surviving quorum of honest
parties}\label{share-recovery-quorum}}

If a quorum of honest parties survives the disastrous event, share
recovery can be accomplished based on peer-to-peer decentralized
identity verification.

Share recovery for a party \(P_i\) with associated decentralized
identity \(I_i\) with address \(pk_i\) and secret key \(sk_i\) proceeds
as follows:

\begin{enumerate}
\def\labelenumi{\arabic{enumi}.}
\tightlist
\item
  For \(P_i\), Initiate a ``share-recovery'' request by following the
  protocol in
  \protect\hyperlink{identity-authed-request-initiation}{section 7.1}.
\item
  For all \(P_j\) where \(j \neq i\):

  \begin{itemize}
  \tightlist
  \item
    Verify the ``share-recovery'' request by following the protocol in
    \protect\hyperlink{identity-authed-request-verification}{section
    7.2}.
  \item
    Initiate an identity challenge for \(P_i\) by following the protocol
    in \protect\hyperlink{identity-challenge-initiation}{section 8.1}.
  \end{itemize}
\item
  For \(P_i\), respond to the identity challenge by following the
  protocol in \protect\hyperlink{identity-challenge-response}{section
  8.2}.
\item
  For all \(P_j\) where \(j \neq i\), verify the identity challenge
  response from \(P_i\) by following the protocol in
  \protect\hyperlink{identity-challenge-verification}{section 8.3}.
\item
  Follow the key refresh protocol described in
  \protect\hyperlink{key-refresh}{section 5} if all verifications above
  pass.
\end{enumerate}

\hypertarget{share-recovery-backup}{%
\subsubsection{13.2. Share recovery with a backup on user-controlled
secondary or device-independent storage}\label{share-recovery-backup}}

\hypertarget{share-recovery-backup-overview}{%
\paragraph{13.2.1. Overview of share recovery with a
backup}\label{share-recovery-backup-overview}}

From the share splitting and reconstruction protocol in
\protect\hyperlink{share-splitting-and-reconstruction}{section 3}, we
note that for any party \(P\), the combination of a signing share \(m\)
and a sub-share \(\beta\) alone is insufficient to reconstruct the
secret share \(x\). This is because a signature of \(m\) from the
decentralized identity \(I\) is required to compute the sub-share
\(\alpha\), so that \(\alpha\) and \(\beta\) can then be used to
reconstruct \(L\) and compute the secret share \(x\) as the constant
term of \(L\).

Therefore, a signing share \(m\) and sub-share \(\beta\) pair can be
safely backed up to user-controlled secondary (e.g.~a secondary device
or a flash drive) or device-independent storage (e.g.~Apple iCloud
\footnote{Apple iCloud. \url{https://www.icloud.com}.}, Google Drive
\footnote{Google Drive. \url{https://drive.google.com}.}, Microsoft
OneDrive \footnote{Microsoft OneDrive.
  \url{https://www.microsoft.com/en-us/microsoft-365/onedrive/online-cloud-storage}.},
Dropbox \footnote{Dropbox. \url{https://www.dropbox.com}.} e.t.c)
without exposing the secret share \(x\).

\hypertarget{share-recovery-backup-encrypt}{%
\paragraph{13.2.2. Generating an encrypted backup for share
recovery}\label{share-recovery-backup-encrypt}}

For increased security, a signature of a standardized phrase can be used
as entropy for generating an encryption secret which can then be used to
encrypt the signing share \(m\) and the sub-share \(\beta\) using a
symmetric encryption algorithm before saving them to back up storage.

Given a standardized phase \(k\), a key derivation function \(H\), a
symmetric encryption algorithm \(E\), this proceeds as follows:

\begin{enumerate}
\def\labelenumi{\arabic{enumi}.}
\tightlist
\item
  Compute the signature \(\phi = Sig(sk, k)\).
\item
  Generate the encryption secret \(\varepsilon = H(\phi)\).
\item
  Compute the ciphertext for the signing share \(m\) as
  \(m_c = E_{enc}(m, \varepsilon)\).
\item
  Compute the ciphertext for the sub-share \(\beta\) as
  \(\beta _c = E_{enc}(\beta, \varepsilon)\).
\item
  Erase both \(\phi\) and \(\varepsilon\) from memory.
\item
  Save \(m_c\) and \(\beta _c\) to backup storage.
\end{enumerate}

\hypertarget{share-recovery-backup-decrypt}{%
\paragraph{13.2.3. Decrypting an encrypted
backup}\label{share-recovery-backup-decrypt}}

Share recovery would then start by signing this standardized phrase,
using the signature to recreate the encryption secret and then
decrypting the encrypted backup to retrieve the signing share \(m\) and
the sub-share \(\beta\).

Given a standardized phase \(k\), a key derivation function \(H\), a
symmetric encryption algorithm \(E\), the ciphertext for the signing
share \(m_c\) and the ciphertext for the sub-share \(\beta _c\), this
proceeds as follows:

\begin{enumerate}
\def\labelenumi{\arabic{enumi}.}
\tightlist
\item
  Compute the signature \(\phi = Sig(sk, k)\).
\item
  Generate the encryption secret \(\varepsilon = H(\phi)\).
\item
  Compute the signing share \(m = E_{dec}(m_c, \varepsilon)\).
\item
  Compute the sub-share \(\beta = E_{dec}(\beta _c, \varepsilon)\).
\item
  Erase both \(\phi\) and \(\varepsilon\) from memory.
\item
  Return the signing share \(m\) and the sub-share \(\beta\).
\end{enumerate}

\hypertarget{share-recovery-backup-enhancements}{%
\paragraph{13.2.4. Further security and usability considerations for
share recovery with a backup}\label{share-recovery-backup-enhancements}}

For further improved security and usability, the signing share \(m\) can
be prefixed with a custom message that alerts the user to the purpose of
the signature. This can help reduce the effectiveness of an adversary
that gains access to the backup and tries to trick the user into signing
\(m\).

Additionally, it's possible to rerun the share splitting protocol to
generate a new pair of a signing share \(m^ \ast\) and a sub-share
\(\beta ^ \ast\) such that \(m^ \ast \neq m\),
\(\beta ^ \ast \neq \beta\) and \(L^ \ast \neq L\) to be specifically
used for backup and recovery. This gives us the option to have separate
signing shares for backup and recovery with customized prefixes that
make it clear to the user that they're signing a backup signing share.

Lastly, the ``backup'' signing share \(m^ \ast\) can be generated based
on user input (e.g.~a passphrase or security questions) removing the
need for it to be backed up together with a sub-share \(\beta ^ \ast\)
but instead relying on the user to provide this input during recovery as
a security-usability tradeoff.

\hypertarget{acknowledgements}{%
\subsection{14. Acknowledgements}\label{acknowledgements}}

This work is funded by a grant from the Ethereum Foundation \footnote{Ethereum
  Foundation: Ecosystem Support Program.
  \url{https://esp.ethereum.foundation}.}.

\hypertarget{references}{%
\subsection{15. References}\label{references}}

\hypertarget{refs}{}
\begin{CSLReferences}{0}{0}
\leavevmode\vadjust pre{\hypertarget{ref-cggmp20}{}}%
\CSLLeftMargin{{[}1{]} }%
\CSLRightInline{Canetti, R., Gennaro, R., Goldfeder, S., Makriyannis, N.
and Peled, U. 2020. UC non-interactive, proactive, threshold ECDSA with
identifiable aborts. \emph{Proceedings of the 2020 ACM SIGSAC conference
on computer and communications security} (New York, NY, USA, 2020),
1769--1787. \url{https://eprint.iacr.org/2021/060}.}

\leavevmode\vadjust pre{\hypertarget{ref-gg20}{}}%
\CSLLeftMargin{{[}2{]} }%
\CSLRightInline{Gennaro, R. and Goldfeder, S. 2020. One round threshold
ECDSA with identifiable abort. Cryptology ePrint Archive, Paper
2020/540. \url{https://eprint.iacr.org/2020/540}.}

\leavevmode\vadjust pre{\hypertarget{ref-cmp20}{}}%
\CSLLeftMargin{{[}3{]} }%
\CSLRightInline{Canetti, R., Makriyannis, N. and Peled, U. 2020. UC
non-interactive, proactive, threshold ECDSA. Cryptology ePrint Archive,
Paper 2020/492. \url{https://eprint.iacr.org/2020/492}.}

\leavevmode\vadjust pre{\hypertarget{ref-sss79}{}}%
\CSLLeftMargin{{[}4{]} }%
\CSLRightInline{Shamir, A. 1979. How to share a secret. \emph{Commun.
ACM}. 22, 11 (Nov. 1979), 612--613.
DOI:https://doi.org/\href{https://doi.org/10.1145/359168.359176}{10.1145/359168.359176}.}

\leavevmode\vadjust pre{\hypertarget{ref-wiki:interpolation}{}}%
\CSLLeftMargin{{[}5{]} }%
\CSLRightInline{Wikipedia. Polynomial interpolation:
\href{https://en.wikipedia.org/wiki/Polynomial_interpolation}{\emph{https://en.wikipedia.org/wiki/Polynomial\_interpolation}}.
Accessed: 2023-05-12.}

\leavevmode\vadjust pre{\hypertarget{ref-gg18}{}}%
\CSLLeftMargin{{[}6{]} }%
\CSLRightInline{Gennaro, R. and Goldfeder, S. 2018. Fast multiparty
threshold ECDSA with fast trustless setup. \emph{Proceedings of the 2018
ACM SIGSAC conference on computer and communications security} (New
York, NY, USA, 2018), 1179--1194.
\url{https://doi.org/10.1145/3243734.3243859}.}

\leavevmode\vadjust pre{\hypertarget{ref-feldman-vss}{}}%
\CSLLeftMargin{{[}7{]} }%
\CSLRightInline{Feldman, P. 1987. A practical scheme for non-interactive
verifiable secret sharing. \emph{Proceedings of the 28th annual
symposium on foundations of computer science} (USA, 1987), 427--438.
\url{https://doi.org/10.1109/SFCS.1987.4}.}

\end{CSLReferences}

\end{document}
